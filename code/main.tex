\documentclass{article}

\usepackage{setspace}

\title{Computational model of lambda calculus}
\date{2018-9-28}
\author{Martin Stoev, Anton Dudov}
\setcounter{tocdepth}{1}

\begin{filecontents}{references.bib}
@BOOK{
	COM_COOPER,
	AUTHOR="S. Barry Cooper",
	TITLE="Computability Theory",
	YEAR="2003",
}
@BOOK{
	COM_ODIFREDDI,
	AUTHOR="Piergiorgio Odifreddi",
	TITLE="Classical recursion theory",
	YEAR="1989",
}
\end{filecontents}

\begin{document}
	\maketitle
	\pagenumbering{gobble}

	\newpage
	\doublespacing
	\tableofcontents
	\singlespacing

	\newpage
	\pagenumbering{arabic}

	\section{Definitions}
		%\subsection{DEFINITION 1.0.0}
		\quad 
		(1) If A is a r.e. set then $\psi_A$ is the enumeration operator defined by it, namely
		\begin{equation}
			x \in \psi^B_A \iff (\exists u \: finite) ((x, u) \in A \land D_u \subseteq B)
		\end{equation}
		\quad
		(2) If $\theta$ is an enumeration operator then $G_\theta$ is a well-defined r.e. set defining it, namely
		\begin{equation}
			(x, u) \in G_\theta \iff x \in \theta^{D_u}
		\end{equation}

	\subsection{Subsectioon}
		easy peasy


	Random citation \cite{COM_COOPER} embeddeed in text.
	Random citation \cite{COM_ODIFREDDI} embeddeed in text.

	\newpage
	\bibliography{references}
	\bibliographystyle{ieeetr}
\end{document}
