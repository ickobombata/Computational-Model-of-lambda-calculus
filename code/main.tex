\documentclass{article}

\usepackage{setspace}
\usepackage{amsfonts}
\usepackage{stmaryrd} % for the square double brackets
\usepackage[utf8]{inputenc} % added with the Theorems and lemmas
\usepackage[english]{babel} % added with the Theorems and lemmas
\usepackage{amsthm} % proofs 

\newtheorem{theorem}{Theorem}[section] % theorems
\newtheorem{lemma}[theorem]{Lemma} % lemmas 
\newtheorem{definition}{Definition}[section] % definitions


\title{Computational model of lambda calculus}
\date{2018-9-28}
\author{Martin Stoev, Anton Dudov}

\setcounter{tocdepth}{1}


\begin{filecontents}{references.bib}
@BOOK{
	COM_COOPER,
	AUTHOR="S. Barry Cooper",
	TITLE="Computability Theory",
	YEAR="2003",
}
@BOOK{
	COM_ODIFREDDI,
	AUTHOR="Piergiorgio Odifreddi",
	TITLE="Classical recursion theory",
	YEAR="1989",
}
\end{filecontents}

\begin{document}
	\maketitle
	\pagenumbering{gobble}

	\newpage
	\doublespacing
	\tableofcontents
	\singlespacing

	\newpage
	\pagenumbering{arabic}

	\section{General Definitions}
		\begin{definition}
			An enumeration operator (or e-operator) $\psi^A$ is a r.e. set. For any A $\subseteq$ $\mathbb{N}$
			\begin{equation}
				x \in \psi^A \iff \exists u \:(finite \: D_u \subseteq A) \and ((x, u) \in \psi)
			\end{equation}
		\end{definition}
			
		\begin{definition}
			If A is a r.e. set then $\psi_A$ is the enumeration operator defined by it, namely
			\begin{equation}
				x \in \psi^B_A \iff \exists u \: (D_u is finite) \and ((x, u) \in A \land D_u \subseteq B)
			\end{equation}
		\end{definition}
		
		\begin{definition}
			If $\theta$ is an enumeration operator then $G_\theta$ is a well-defined r.e. set defining it, namely
			\begin{equation}
				(x, u) \in G_\theta \iff x \in \theta^{D_u}
			\end{equation}
		\end{definition}

	\begin{lemma}
		If $\psi$ is an enumeration operator, then $\psi_{G_{\psi}} = \psi$
	\end{lemma}	
	\begin{proof}
		TO BE DONE @anton
	\end{proof}


	\begin{definition}
		Let $\eta$ be an assigment of r.e. sets to the variables of 
		lambda calculus. With every $\lambda$-term E we inductively 
		associate a r.e. set $\llbracket E \rrbracket_{\eta}$:
			\begin{enumerate}
				\item $\llbracket x \rrbracket_{\eta} = \eta (x)$
				\item $\llbracket E_1 E_2 \rrbracket_{\eta} = 
					\psi_{\llbracket E_1 \rrbracket_\eta} 
					(\llbracket E_2 \rrbracket_\eta)$
				\item $\llbracket \lambda x.E \rrbracket_{\eta} = 
					G_{\lambda X. \llbracket E \rrbracket_{\eta [x := X]}}$
			\end{enumerate}
			Where $\lambda X. \llbracket E \rrbracket_{\eta [x := X]}$ is a function

		\begin{equation}
			A \in W \mapsto \llbracket E \rrbracket_{\eta [x := A]}
		\end{equation}
	\end{definition}

	Random citation \cite{COM_COOPER} embeddeed in text.
	Random citation \cite{COM_ODIFREDDI} embeddeed in text.

	\newpage
	\bibliography{references}
	\bibliographystyle{ieeetr}
\end{document}
