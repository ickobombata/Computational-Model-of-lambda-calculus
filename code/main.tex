\documentclass{article}

\usepackage{setspace}
\usepackage{amsfonts}
\usepackage{stmaryrd} % for the square double brackets
\usepackage[utf8]{inputenc} % added with the Theorems and lemmas
\usepackage[english]{babel} % added with the Theorems and lemmas
\usepackage{amsthm} % proofs 
\usepackage{amssymb} % two head arrow
\usepackage{mathtools, amsmath} % multiline

\newtheorem{theorem}{Theorem}[section] % theorems
\newtheorem{lemma}[theorem]{Lemma} % lemmas 
\newtheorem{definition}{Definition}[section] % definitions


\title{Computational model of lambda calculus}
\date{2018-9-28}
\author{Martin Stoev, Anton Dudov}

\setcounter{tocdepth}{1}


\begin{filecontents}{references.bib}
@BOOK{
	COM_COOPER,
	AUTHOR="S. Barry Cooper",
	TITLE="Computability Theory",
	YEAR="2003",
}
@BOOK{
	COM_ODIFREDDI,
	AUTHOR="Piergiorgio Odifreddi",
	TITLE="Classical recursion theory",
	YEAR="1989",
}
\end{filecontents}

\begin{document}
	\maketitle
	\pagenumbering{gobble}

	\newpage
	\doublespacing
	\tableofcontents
	\singlespacing

	\newpage
	\pagenumbering{arabic}

	\section{General Definitions}
		\begin{definition}
			An enumeration operator (or e-operator) $\Psi^A$ is a r.e. set. For any A $\subseteq$ $\mathbb{N}$
			\begin{equation}
				x \in \Psi^A \iff \exists u \:(finite \: D_u \subseteq A) \and ((x, u) \in \Psi)
			\end{equation}
		\end{definition}
			
		\begin{definition}
			If A is a r.e. set then $\Psi_A$ is the enumeration operator defined by it, namely
			\begin{equation}
				x \in \Psi^B_A \iff \exists u \: (D_u is finite) \and ((x, u) \in A \land D_u \subseteq B)
			\end{equation}
		\end{definition}
		
		\begin{definition}
			If $\theta$ is an enumeration operator then $G_\theta$ is a well-defined r.e. set defining it, namely
			\begin{equation}
				(x, u) \in G_\theta \iff x \in \theta^{D_u}
			\end{equation}
		\end{definition}

	\begin{lemma}
		If $\Psi$ is an enumeration operator, then $\Psi_{G_{\Psi}} = \Psi$
	\end{lemma}	
	\begin{proof}
		TO BE DONE @anton
	\end{proof}


	\begin{definition}
		Let $\eta$ be an assignment of r.e. sets to the variables of 
		lambda calculus. With every $\lambda$-term E we inductively 
		associate a r.e. set $\llbracket E \rrbracket_{\eta}$:
			\begin{enumerate}
				\item $\llbracket x \rrbracket_{\eta} = \eta (x)$
				\item $\llbracket E_1 E_2 \rrbracket_{\eta} = 
					\Psi_{\llbracket E_1 \rrbracket_\eta} 
					(\llbracket E_2 \rrbracket_\eta)$
				\item $\llbracket \lambda x.E \rrbracket_{\eta} = 
					G_{\lambda X. \llbracket E \rrbracket_{\eta [x := X]}}$
			\end{enumerate}
			Where $\lambda X. \llbracket E \rrbracket_{\eta [x := X]}$ is a function

		\begin{equation}
			A \in W \mapsto \llbracket E \rrbracket_{\eta [x := A]}
		\end{equation}
	\end{definition}

	\begin{lemma}
		For any environment $\eta$ and term t, 
			$\llbracket$t$\rrbracket_\eta$ is an c.e. set.
	\end{lemma}
	\begin{proof}
		By structural induction on the definition of 
			$\llbracket$t$\rrbracket_\eta$.
		\begin{enumerate}
			\item 
				$\llbracket x \rrbracket_\eta = \eta (x)$ by definition
			\item
				To show that $\Psi_{\llbracket E_1 \rrbracket_\eta} 
				(\llbracket E_2 \rrbracket_\eta)$ is a c.e. set we
				prove that $\Psi^B_A$ is an enumeration operator which 
				follows from 
				\begin{equation} 
					n \in \Psi^B_A \iff \exists u (D_u \subseteq B 
					\and <n, u> \in A)
				\end{equation}
			\item
				To be done.
		\end{enumerate}
	\end{proof}
	
	\begin{lemma}
		For the following theorem we will need one lemma
		beforehand for better readability, namely
		\begin{equation}
			\llbracket u \rrbracket_{\eta [x := 
				\llbracket v \rrbracket_{\eta}]} = 
				\llbracket u [x \mapsto v] \rrbracket_{\eta}
		\end{equation}
	\end{lemma}
	\begin{proof}
		Structural induction on u.
		\begin{enumerate}
			\item u = x, then:
			\begin{equation}
				\llbracket x \rrbracket_{\eta [x := 
				\llbracket v \rrbracket_{\eta}]} = 
				\eta (x) =
				\llbracket v \rrbracket_{\eta} =
				\llbracket x [x \mapsto v] \rrbracket_{\eta}
			\end{equation}
			
			\item u = y $\neq$ x, then:
			\begin{equation}
				\llbracket y \rrbracket_{\eta [x := 
				\llbracket v \rrbracket_{\eta}]} = 
				\eta (y) =
				\llbracket y \rrbracket_{\eta} =
				\llbracket y [x \mapsto v] \rrbracket_{\eta}
			\end{equation}

			\item u = pq, then:
			\begin{equation*}
			\begin{split}
				\llbracket pq \rrbracket_{\eta [x := 
				\llbracket v \rrbracket_{\eta}]}
 				&\stackrel{def 1.4.2}{=}
				\Psi_{\llbracket p \rrbracket_{
					\eta [x := \llbracket v \rrbracket_{\eta}}]}
					(\llbracket q \rrbracket_{
					\eta [x:= \llbracket v \rrbracket_{\eta}]}) \\
 				&\stackrel{ind.hyp.}{=}
				\Psi_{\llbracket p[x \mapsto v] \rrbracket_{\eta}
					(\llbracket q[x \mapsto v] \rrbracket_{\eta}]}) \\
 				&\stackrel{def 1.4.2}{=}
				\llbracket p[x \mapsto v]q[x \mapsto v] 
					\rrbracket_{\eta}\\
 				&\stackrel{def App}{=}
				\llbracket pq [x \mapsto v] \rrbracket_{\eta}
			\end{split}
			\end{equation*}

			\item u = $\lambda_y$ p, then:
			\begin{equation*}
			\begin{split}
				\llbracket \lambda_y p \rrbracket_{\eta [x := 
				\llbracket v \rrbracket_{\eta}]}
 				&\stackrel{def 1.4.3}{=}
				G_{\Lambda Y. \llbracket p \rrbracket_{
					\eta [x :] \llbracket v \rrbracket_{\eta} ; 
						y := Y]}} \\
 				&\stackrel{ind.hyp}{=}
				G_{\Lambda Y. \llbracket p[x \mapsto v] \rrbracket_{
					\eta[ y:=Y]} } \\
 				&\stackrel{def 1.4.3}{=}
				\llbracket \lambda_y p[x \mapsto v]\rrbracket_{\eta}
			\end{split}
			\end{equation*}
		\end{enumerate}
	\end{proof}

	\begin{theorem}
		If $E_1 \stackrel{\beta}{\twoheadrightarrow}  E_2$ then 
			$\llbracket E_1
		\rrbracket_{\eta} = \llbracket E_2 \rrbracket_{\eta}$
		for any $\eta$.
	\end{theorem}
	\begin{proof}
		We will prove one step of the $\beta$ reduction and then
		by induction the rest will follow

		We have that $(\lambda x E_1) E_2 = E_1 [x \mapsto E_2]$
		and we will prove that 

		$\llbracket (\lambda x E_1) E_2
		\rrbracket_{\eta} = \llbracket E_1 [x \mapsto E_2] 
		\rrbracket_{\eta}$
		\begin{equation*}
     \begin{split}
			\llbracket (\lambda x E_1) E_2 \rrbracket_{\eta}
			&\:\,\stackrel{def. 1.4.2}{=} \:
			\Psi_{\llbracket \lambda x E_1 \rrbracket_{\eta}}
				(\llbracket E_2 \rrbracket_{\eta}) \\
			&\:\,\stackrel{def. 1.4.3}{=} \:
			\Psi_{G_{\Lambda X \llbracket E_1 \rrbracket_{
				\eta [x := X]}}}
				(\llbracket E_2 \rrbracket_{\eta}) \\
			&\stackrel{Lemma 1.1}{=}
				\Lambda X \llbracket E_1 \rrbracket_{\eta [x := X]}
				(\llbracket E_2 \rrbracket_{\eta}) \\
			&\;\;\:\:\stackrel{def \Lambda}{=}\:\:\:\:\:
				\llbracket E_1 \rrbracket_
					{\eta [x := \llbracket E_2 \rrbracket_{\eta}]} \\
			&\stackrel{Lemma 1.3}{=}\:
				\llbracket E_1 [ x \mapsto E_2 ] \rrbracket_{\eta}
     \end{split}
		\end{equation*}
	\end{proof}

	Random citation \cite{COM_COOPER} embeddeed in text.
	Random citation \cite{COM_ODIFREDDI} embeddeed in text.

	\newpage
	\bibliography{references}
	\bibliographystyle{ieeetr}
\end{document}

















