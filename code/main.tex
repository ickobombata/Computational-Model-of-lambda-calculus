\documentclass{article}

\usepackage{setspace}
\usepackage{amsfonts}
\usepackage{stmaryrd} % for the square double brackets
\usepackage[utf8]{inputenc} % added with the Theorems and lemmas
\usepackage[english]{babel} % added with the Theorems and lemmas
\usepackage{amsthm} % proofs 
\usepackage{amssymb} % two head arrow
\usepackage{mathtools, amsmath} % multiline

\newtheorem{theorem}{Theorem}[section] % theorems
\newtheorem{lemma}[theorem]{Lemma} % lemmas 
\newtheorem{definition}{Definition}[section] % definitions
\newtheorem{exercise}{Exercise}[section]

\title{Computational model of lambda calculus}
\date{2018-9-28}
\author{Martin Stoev, Anton Dudov}

\setcounter{tocdepth}{1}


\begin{filecontents}{references.bib}
@BOOK{
	COM_COOPER,
	AUTHOR="S. Barry Cooper",
	TITLE="Computability Theory",
	YEAR="2003",
}
@BOOK{
	COM_ODIFREDDI,
	AUTHOR="Piergiorgio Odifreddi",
	TITLE="Classical recursion theory",
	YEAR="1989",
}
\end{filecontents}

\begin{document}
	\maketitle
	\pagenumbering{gobble}

	\newpage
	\doublespacing
	\tableofcontents
	\singlespacing

	\newpage
	\pagenumbering{arabic}

	\section{General Definitions}
		\begin{definition}
			An enumeration operator (or e-operator) $\Psi^A$ is a r.e. set. For any A $\subseteq$ $\mathbb{N}$
			\begin{equation}
				x \in \Psi^A \iff \exists u \:(finite \: D_u \subseteq A) \:\&\: ((x, u) \in \Psi)
			\end{equation}
		\end{definition}
			
		\begin{definition}
			If A is a r.e. set then $\Psi_A$ is the enumeration operator defined by it, namely
			\begin{equation}
				x \in \Psi^B_A \iff \exists u \: (D_u is finite) \:\&\: ((x, u) \in A \land D_u \subseteq B)
			\end{equation}
		\end{definition}
		
		\begin{definition}
			If $\theta$ is an enumeration operator then $G_\theta$ is a well-defined r.e. set defining it, namely
			\begin{equation}
				(x, u) \in G_\theta \iff x \in \theta^{D_u}
			\end{equation}
		\end{definition}

	\begin{lemma}
		If $\Phi$ is an enumeration operator, then $\Phi_{G_{\Psi}} = \Psi$
	\end{lemma}	
	\begin{proof}
		($\subseteq$)
		\begin{equation*}
		\begin{split}
			x \in \Phi_{G_\Psi}^B &\Rightarrow \exists u \: 
				<x, u> \in G_\Psi \:\&\: D_u \subseteq B\\
			&\Rightarrow \exists u \: 
				<x, u> \in \Psi^{D_u} \:\&\: D_u \subseteq B 
		\end{split}
		\end{equation*}
		\qquad since Du $\subseteq$ B \& e-operator is monotone 
		it follows that: $x \in \Psi^B$
		\newline
		($\supseteq$)
		\newline
		Since $\Psi$ is an e-operator, then 
		$\forall B \: \exists e:\: \Psi = \Phi_e$, therefore:
		\begin{equation*}
		\begin{split}
			x \in \Phi_e^B &\Rightarrow \exists u \: 
				<x, u> \in W_e \:\&\: D_u \subseteq B\\
			&\Rightarrow \exists u \: 
				<x, u> \in W_e \:\&\: D_u \subseteq D_u,\:therefore\\
			&x \in \Psi^{D_u} \:\Rightarrow\: <x, u> \in G_\Psi\\
			&\Rightarrow \exists u \: 
				<x, u> \in G_\Psi \:\&\: D_u \subseteq B\\
			&\Rightarrow x \in \Phi_{G_\Psi}
		\end{split}
		\end{equation*}

	\end{proof}


	\begin{definition}
		Let $\eta$ be an assignment of r.e. sets to the variables of 
		lambda calculus. With every $\lambda$-term E we inductively 
		associate a r.e. set $\llbracket E \rrbracket_{\eta}$:
			\begin{enumerate}
				\item $\llbracket x \rrbracket_{\eta} = \eta (x)$
				\item $\llbracket E_1 E_2 \rrbracket_{\eta} = 
					\Psi_{\llbracket E_1 \rrbracket_\eta} 
					(\llbracket E_2 \rrbracket_\eta)$
				\item $\llbracket \lambda x.E \rrbracket_{\eta} = 
					G_{\lambda X. \llbracket E \rrbracket_{\eta [x := X]}}$
			\end{enumerate}
			Where $\lambda X. \llbracket E \rrbracket_{\eta [x := X]}$ is a function

		\begin{equation}
			A \in W \mapsto \llbracket E \rrbracket_{\eta [x := A]}
		\end{equation}
	\end{definition}

	\begin{lemma}
		For any environment $\eta$ and term t, 
			$\llbracket$t$\rrbracket_\eta$ is an c.e. set.
	\end{lemma}
	\begin{proof}
		By structural induction on the definition of 
			$\llbracket$t$\rrbracket_\eta$.
		\begin{enumerate}
			\item 
				$\llbracket x \rrbracket_\eta = \eta (x)$ by definition
			\item
				To show that $\Psi_{\llbracket E_1 \rrbracket_\eta} 
				(\llbracket E_2 \rrbracket_\eta)$ is a c.e. set we
				prove that $\Psi^B_A$ is an enumeration operator which 
				follows from 
				\begin{equation} 
					n \in \Psi^B_A \iff \exists u (D_u \subseteq B 
					\:\&\: <n, u> \in A)
				\end{equation}
			\item
				Since the graph of an enumeration operator is an r.e.
				set, to show that 
				$\llbracket \lambda_x u \rrbracket_\eta$ 
				$\stackrel{def 1.4.3}{=}$ 
				$G_{\Lambda X. \llbracket u \rrbracket_{\eta [x := X]}}$
				is an r.e. set we prove that
				$\Lambda X. \llbracket u \rrbracket_{\eta [x := X]}$
				is an e-operator. This can be done inductively on u:

 				Let 
				\begin{equation*}
				\begin{split}
					\llbracket E \rrbracket_\eta[x := W_1\; y := W_2]
					= \prescript{E}{}{\Psi}{}{}^W_1 \oplus W_2, where\\
					\prescript{E}{}{\Psi}{}{} = \{<x, v> |\:x \in 
					\llbracket E \rrbracket_{\eta_{
					[x := L(D_v)\; y := R(D_v)]}}\}\\
					x \in \prescript{E}{}{\Psi}{}{}^W \iff \exists u \: 
					<x, u>\:\in\:\prescript{E}{}{\Psi}{}{}\:\&\: 
					D_u \subseteq W
				\end{split}
				\end{equation*}
				\begin{itemize}
  				\item If u = x, then 
					$\Lambda X. \llbracket u \rrbracket_{\eta [x := X]}$
					is the identity function $\Lambda X. X$, which is 
					an enumeration operator.
  				\item If u = y $\neq$ x, then 
					$\Lambda X. \llbracket u \rrbracket_{\eta [x := X]}$
					is the constant function $\Lambda X. \eta(y)$, which
					is an enumeration operator.
  				\item If u = $u_1 u_2$, then
					$\Lambda X. \llbracket u \rrbracket_{\eta [x := X]}$
					is $\Lambda X \Phi_{\llbracket u_1 \rrbracket_{\eta 
					[x := X]}} (\llbracket u_2 \rrbracket_{\eta 
					[x := X]})$. Where by the i.h. 
					$\llbracket u_1 \rrbracket_{\eta [x := X]}$
					is an e operator, namely
					$\prescript{u_1}{}{\Psi}{}{}$
					and 
					$\llbracket u_2 \rrbracket_{\eta [x := X]}$
					is an e operator, namely 
					$\prescript{u_2}{}{\Psi}{}{}$
					Therefore $\Phi_{\prescript{u_1}{}{\Psi}{}{}} 										(\prescript{u_2}{}{\Psi}{}{})$ is an e operator
					(by the proof from 1.2.2).
  				\item If u = $\lambda_y$ E.
				Then
				\begin{equation*}
				\begin{split}
					\llbracket \lambda_y E \rrbracket_\eta{ [x := W_i]}
					\stackrel{def 1.4.3}{=}
					G_{\Lambda Y. \llbracket E \rrbracket_\eta
					{[x := W_i, y:= Y]}}
					\\{=}
					\{<x, v> | x \in 
					\llbracket E \rrbracket_\eta {[x := W_i, y:= D_v]}\}
					\\\stackrel{i.h.}{=}
					\{<x, v> | x \in 
					\prescript{E}{}{\Psi}{}{} \: W \oplus\ D_v\}
					\\{=}
					\{<x, v> | \exists u \:	<x,\:u>\:\in\:
					\prescript{E}{}{\Psi}{}{}\:\&\: 
					D_u \subseteq W \oplus D_v\}
					\\\stackrel{def \prescript{E}{}{\Psi}{}{}}{=}
					\{<x, v> | \exists u \: 	x\:\in\:
					\llbracket E \rrbracket_\eta
					{[x := L(D_u), y:= R(D_u)]}
					\:\&\: 	D_u \subseteq W \oplus D_v\}	
					\\{=}
					\{<<x,\:v>,\:u> | x \:\in\:
					\llbracket E \rrbracket_\eta
					{[x := L(D_u), y:= R(D_u)]}
					\:\&\: 	D_u \subseteq W \oplus D_v\}
				\end{split}
				\end{equation*}
				Since $\llbracket E \rrbracket_\eta
				{[x := L(D_u), y:= R(D_u)]}$ is an e operator(i.h)
				(e operators are r.e. by definition) and
				$D_u \subseteq W \oplus D_v$ is r.e. set
				it follows that the above set is an e operator.
				\end{itemize}
		\end{enumerate}
	\end{proof}
	
	\begin{lemma}
		For the following theorem we will need one lemma
		beforehand for better readability, namely
		\begin{equation}
			\llbracket u \rrbracket_{\eta [x := 
				\llbracket v \rrbracket_{\eta}]} = 
				\llbracket u [x \mapsto v] \rrbracket_{\eta}
		\end{equation}
	\end{lemma}
	\begin{proof}
		Structural induction on u.
		\begin{enumerate}
			\item u = x, then:
			\begin{equation}
				\llbracket x \rrbracket_{\eta [x := 
				\llbracket v \rrbracket_{\eta}]} = 
				\eta (x) =
				\llbracket v \rrbracket_{\eta} =
				\llbracket x [x \mapsto v] \rrbracket_{\eta}
			\end{equation}
			
			\item u = y $\neq$ x, then:
			\begin{equation}
				\llbracket y \rrbracket_{\eta [x := 
				\llbracket v \rrbracket_{\eta}]} = 
				\eta (y) =
				\llbracket y \rrbracket_{\eta} =
				\llbracket y [x \mapsto v] \rrbracket_{\eta}
			\end{equation}

			\item u = pq, then:
			\begin{equation*}
			\begin{split}
				\llbracket pq \rrbracket_{\eta [x := 
				\llbracket v \rrbracket_{\eta}]}
 				&\stackrel{def 1.4.2}{=}
				\Psi_{\llbracket p \rrbracket_{
					\eta [x := \llbracket v \rrbracket_{\eta}}]}
					(\llbracket q \rrbracket_{
					\eta [x:= \llbracket v \rrbracket_{\eta}]}) \\
 				&\stackrel{ind.hyp.}{=}
				\Psi_{\llbracket p[x \mapsto v] \rrbracket_{\eta}
					(\llbracket q[x \mapsto v] \rrbracket_{\eta}]}) \\
 				&\stackrel{def 1.4.2}{=}
				\llbracket p[x \mapsto v]q[x \mapsto v] 
					\rrbracket_{\eta}\\
 				&\stackrel{def App}{=}
				\llbracket pq [x \mapsto v] \rrbracket_{\eta}
			\end{split}
			\end{equation*}

			\item u = $\lambda_y$ p, then:
			\begin{equation*}
			\begin{split}
				\llbracket \lambda_y p \rrbracket_{\eta [x := 
				\llbracket v \rrbracket_{\eta}]}
 				&\stackrel{def 1.4.3}{=}
				G_{\Lambda Y. \llbracket p \rrbracket_{
					\eta [x :] \llbracket v \rrbracket_{\eta} ; 
						y := Y]}} \\
 				&\stackrel{ind.hyp}{=}
				G_{\Lambda Y. \llbracket p[x \mapsto v] \rrbracket_{
					\eta[ y:=Y]} } \\
 				&\stackrel{def 1.4.3}{=}
				\llbracket \lambda_y p[x \mapsto v]\rrbracket_{\eta}
			\end{split}
			\end{equation*}
		\end{enumerate}
	\end{proof}

	\begin{theorem}
		If $E_1 \stackrel{\beta}{\twoheadrightarrow}  E_2$ then 
			$\llbracket E_1
		\rrbracket_{\eta} = \llbracket E_2 \rrbracket_{\eta}$
		for any $\eta$.
	\end{theorem}
	\begin{proof}
		We will prove one step of the $\beta$ reduction and then
		by induction the rest will follow

		We have that $(\lambda x E_1) E_2 = E_1 [x \mapsto E_2]$
		and we will prove that 

		$\llbracket (\lambda x E_1) E_2
		\rrbracket_{\eta} = \llbracket E_1 [x \mapsto E_2] 
		\rrbracket_{\eta}$
		\begin{equation*}
     \begin{split}
			\llbracket (\lambda x E_1) E_2 \rrbracket_{\eta}
			&\:\,\stackrel{def. 1.4.2}{=} \:
			\Psi_{\llbracket \lambda x E_1 \rrbracket_{\eta}}
				(\llbracket E_2 \rrbracket_{\eta}) \\
			&\:\,\stackrel{def. 1.4.3}{=} \:
			\Psi_{G_{\Lambda X \llbracket E_1 \rrbracket_{
				\eta [x := X]}}}
				(\llbracket E_2 \rrbracket_{\eta}) \\
			&\stackrel{Lemma 1.1}{=}
				\Lambda X \llbracket E_1 \rrbracket_{\eta [x := X]}
				(\llbracket E_2 \rrbracket_{\eta}) \\
			&\;\;\:\:\stackrel{def \Lambda}{=}\:\:\:\:\:
				\llbracket E_1 \rrbracket_
					{\eta [x := \llbracket E_2 \rrbracket_{\eta}]} \\
			&\stackrel{Lemma 1.3}{=}\:
				\llbracket E_1 [ x \mapsto E_2 ] \rrbracket_{\eta}
     \end{split}
		\end{equation*}
	\end{proof}

	\begin{definition}
		We define the fixed point combinator Y to be:
		\begin{equation}
			Y \stackrel{def}{=} \lambda f.
				((\lambda x. f(x\:x))\:
				((\lambda x. f(x\:x))
		\end{equation}
	\end{definition}
	\begin{exercise}
		Show that any $\lambda$-term F has fixed point YF, such that
			\begin{equation}
				YF\:=\:F(YF).
			\end{equation}
	\end{exercise}
	\begin{proof}
		\begin{equation*}
     \begin{split}
			YF &= ((\lambda x. F(x\:x))\:((\lambda x. F(x\:x))\\
				&= F((\lambda x. F(x\:x))\:((\lambda x. F(x\:x))\\
				&= F(\lambda f. 
						((\lambda x. f(x\:x))\:
						((\lambda x. f(x\:x)) F)\\
				&= F(YF)
     \end{split}
		\end{equation*}
	\end{proof}

	\begin{exercise}
		Let $\Psi$ be any e-operator. Show that $\Psi$ can be
		expressed as:
			\begin{equation}
				\Psi\:=\:\Psi_{\llbracket x \rrbracket_{\eta}}
			\end{equation}
		for suitable choise of assigment $\eta$.
	\end{exercise}
	\begin{proof}
		Since $\Psi$ is an e-operator, then 
		$\forall B \: \exists e:\: \Psi = \Phi_e$, therefore:
		\begin{equation*}
		\begin{split}
			n \in \Psi^B &\iff \exists u \: 
				<n, u> \in W_e \:\&\: D_u \subseteq B\\
			&\iff \exists u \: <n, u> \in 
				\llbracket x \rrbracket_{\eta} \:\&\: D_u \subseteq B
				,\:\: where \:\:\eta(x) = W_e\\
			&\iff n \in \Psi_{\llbracket x \rrbracket_{\eta}}^B
		\end{split}
		\end{equation*}
	\end{proof}

	\begin{exercise}
		Let $\Psi$ be any e-operator. Show that $\Psi$ has a 
		r.e. fixed point W, such that:
			\begin{equation}
				\Psi^W\:=\:W
			\end{equation}
	\end{exercise}
	\begin{proof}
		Let $W=\llbracket Yx \rrbracket_{\eta}$, then:
		\begin{equation}
			\llbracket Yx \rrbracket_{\eta} =
				\llbracket x(Yx) \rrbracket_{\eta} =
				\Phi_
					{\llbracket x \rrbracket_{\eta}} 
					(\llbracket Yx \rrbracket_{\eta})
		\end{equation}
		\begin{equation*}
		\begin{split}
			n \in \Phi_
					{\llbracket x \rrbracket_{\eta}} 
					(\llbracket Yx \rrbracket_{\eta})
				&\iff \exists u \: <n, u> \in 
				\llbracket x \rrbracket_{\eta} 
					\:\&\: D_u \subseteq \llbracket Yx \rrbracket_{\eta}
					,\:let\:\eta(x) = W_e\\
				&\iff \exists u \: <n, u> \in W_e
					\:\&\: D_u \subseteq \llbracket Yx \rrbracket_{\eta}\\
				&\iff \eta \in \Psi^{\llbracket Yx \rrbracket_{\eta}}
		\end{split}
		\end{equation*}
	\end{proof}

	\begin{exercise}
		Show that for any assigment $\eta$:
			\begin{equation}
				\llbracket Yx \rrbracket_{\eta[x := \emptyset]} 
					= \emptyset
			\end{equation}
	\end{exercise}
	\begin{proof}
		We know that: 
		\begin{equation}
			\llbracket Yx \rrbracket_{\eta [x := \emptyset]} =
				\llbracket x(Yx) \rrbracket_{\eta [x := \emptyset]}\\
			= \Phi_
					{\llbracket x \rrbracket_{\eta[x := \emptyset]}} 
					(\llbracket Yx \rrbracket_{\eta[x := \emptyset]})
		\end{equation}

		\begin{equation*}
		\begin{split}
			n \in \Phi_
					{\llbracket x \rrbracket_{\eta[x := \emptyset]}} 
					(\llbracket Yx \rrbracket_{\eta[x := \emptyset]})
				&\iff \exists u \: <n, u> \in 
				\llbracket x \rrbracket_{\eta[x := \emptyset]} 
					\:\&\: D_u \subseteq 
						\llbracket Yx \rrbracket_{\eta[x := \emptyset]}\\
				&\iff \exists u \: <n, u> \in \emptyset 
					\:\&\: D_u \subseteq 
						\llbracket Yx \rrbracket_{\eta[x := \emptyset]}\\
				&\quad\Rightarrow \neg \exists n\: :\: 
					n \in \Phi_{\llbracket x \rrbracket_{
						\eta[x := \emptyset]}}
		\end{split}
		\end{equation*}
	\end{proof}
	Random citation \cite{COM_COOPER} embeddeed in text.
	Random citation \cite{COM_ODIFREDDI} embeddeed in text.

	\newpage
	\bibliography{references}
	\bibliographystyle{ieeetr}
\end{document}

















